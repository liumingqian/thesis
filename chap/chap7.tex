% Copyright (c) 2014,2016,2018 Casper Ti. Vector
% Public domain.

\chapter{总结与展望}
\section{工作总结}
本文设计与实现了大规模三维场景中地形的实时编辑框架。本文实现的地形编辑器能在虚拟地球环境下对地形的高程、纹理和语义信息进行编辑,并通过统一的编辑操作框架实现了编辑操作的撤销和重做、编辑结果的保存与加载等编辑必备的功能,较为详细的阐述了实现的主要思路和关键细节。同时本文考虑了在分层四叉树上编辑的时间、空间开销及不同层间同步的问题,保证了在大规模场景中编辑的效率和正确性。本文实现的编辑器可以快速构建符合用户创作意图的虚拟场景,在战场模拟等应用场景的环境构建中得到了良好的应用。使用该编辑器可以在10到90分钟内快速构建地形场景。\par
本文对过程式地形进行了探讨,通过过程式的为粗糙地形纹理数据添加细节,有效提升了近地表视角漫游时地形的真实感,丰富了地形场景的内容。并提供了一种过程式编辑方法,通过简单设置过程式编辑规则的参数,使用者可以通过少量模糊的编辑操作,产生精细且符合真实世界地理规律的编辑结果,极大的提升了编辑效率。\par
本文还对地形语义信息对虚拟地形系统的辅助作用进行了探讨。首先给出了一种语义信息的定义方式,并实现了一种从地形高程和纹理数据生成语义数据,并通过手工编辑完善和修改语义数据的方法。考虑到语义信息与地形高程和纹理信息的强相关性,本文实现了一种进行高程编辑时对语义进行联动编辑的方法。最终,给出了语义信息指导海洋和内陆水体绘制的实例,验证了语义信息的实用性和必要性。
\section{下一步工作}
基于粗糙地形的过程式编辑和过程式绘制依然是有很多发掘空间的课题,例如本文的工作只对粗糙数据提取了基本的颜色信息和高程规则信息,如能结合卫星图像的语义分割信息,应当可以为粗糙地形进行的过程式合成提供更多可能性,并可能引入深度学习方法,通过简单勾勒和自动填充等交互方式进一步减少人工干预。\par
目前高程、纹理和语义信息还是分别编辑,当对高程或纹理数据进行编辑时语义数据应当相应的被更新,减少编辑工作。受到算法实现的限制,海洋模块还需要有正确的精细的高程数据,才能配合语义数据进一步提升海岸线的精细程度,协同编辑可以更好的解决此类问题。下一步,语义信息还可以用于指导植被和城市建筑的分布,并在飞行模拟中提供地表摩擦力数据。\par
未来地形编辑器还需承担更多编辑功能,如地物的摆放与编辑、植被的生成、湖泊河流等内陆水体的编辑。发展矢量编辑也是提升编辑器性能的重要途径,包括编辑点、线、多边形等几何元素。通过支持分层四叉树上的矢量编辑,可以提高数据精度,减少编辑信息的内存开销,提升交互速度。

% vim:ts=4:sw=4
