% Copyright (c) 2014,2016,2018 Casper Ti. Vector
% Public domain.

\chapter{引言}
\section{研究背景及意义}
在游戏、电影制作、虚拟战场,飞行模拟等虚拟场景中,地形是重要的组成部分,真实感地形可以提升虚拟场景的沉浸感和可信度。例如,开放式游戏通常构建大规模、高真实感的虚拟地形让玩家去探索;电影中有时需要合成现实中不存在或难以拍摄到的地形场景;虚拟战场、飞行模拟等应用也需要构建各种地形场景用于仿真或训练。在上述应用中,构建虚拟地形场景的第一步是生成带有纹理的地形,以构建表示地形起伏的几何外形。随后,场景设计者将岩石、树木、植物和建筑物等地物放置在地形表面,以增强场景的真实感。不同的虚拟场景对地形有不同的要求,因此虚拟场景构建需要专门的地形编辑器以满足需要。地形编辑器的设计与开发是构建虚拟现实引擎之外的重要工作。\par
在地形编辑器中构造真实感地形是非常有挑战性的工作,因为真实世界中的地形非常复杂多变。从平坦的平原、板块挤压形成的山脉,到被流水侵蚀形成的山谷等,各种类型的地形都可能在局部区域中共存。地形的形成是自然环境中的各种元素长期相互作用的结果,如逐渐侵蚀地形的风力和水力、山体滑坡、被闪电击中等。不同的地形特征可能在不同的尺度上呈现,如冰川和板块运动的影响范围可能跨越数十至数百公里,而某些水力和风力侵蚀效果仅在几米的尺度内显现。一般来说,地形编辑器内可供选择的工具通常包括不同形状的笔刷、直线和曲线样条工具、区域选择工具等等,用户通过选择合适的编辑工具、调整工具相关参数来实现符合不同创作意图的地形。\par

目前的地形编辑器主要有三种类型:游戏开发引擎(如Unreal\supercite{unreal}、CryEngine\supercite{cry-engine}、Unity3D\supercite{unity}等)的一部分;特效编辑引擎(如houdini\supercite{houdini}等)的一部分;独立的地形编辑软件(如Terragen\supercite{terragen}、World Machine\supercite{world-machine}等)。地形编辑器的编辑对象通常为地形高程和地面纹理。一些编辑器如 Unity、Unreal 等,还包括对植被等地理要素和建筑等地物的编辑。Eric等\supercite{eric-review}对地形生成领域的问题进行了分析和归类,指出地形生成的四个最主要的挑战:大规模精细地形的表示、地形材质的应用、地形形状的精细控制和地形的地理真实性。并指出地形高程生成目前有四类主流的解决方案:(1)基于分形噪声的过程式方法\supercite{Belhadj2007Terrain}可以快速生成地形,但缺乏对创作过程的直接控制,且结果缺乏地理学上的真实性。(2)基于样例的方法\supercite{TasseEnhanced}在与深度学习方法结合时可以生成非常逼真的效果,但对于每种特定的地形(山脉、丘陵、河流、高原等)都需要上游的训练阶段。(3)基于物理仿真的方法\supercite{Mei2007Fast}通常基于真实物理世界中的热力学、水力学、地质学等规则对地形进行腐蚀,其迭代过程模拟了真实世界的地形形成过程,因此可以产生非常真实的效果,但缺点是无法让艺术家或设计师直观有效的控制生成过程。(4)基于草图的方法\supercite{Zhou2007Terrain}提供了直观的工具来控制地形形状,但生成的地形往往真实感较差。此外,当编辑区域比较大的时候,手工编辑生成草图将带来非常大的时间开销。\par
近年来,虚拟战场、飞行模拟等应用越来越依赖逼真的地形,并在地形精细程度和规模等方面提出了越来越高的要求,推动了大规模地形引擎的发展。本文基于北京大学图形与交互技术实验室研发的ViWo(Virtual World)系统实现,ViWo系统包含了面向飞行模拟器的视景仿真系统和复杂场景绘制引擎,支持从全球尺度到近地面尺度的三维场景漫游。ViWo系统的地形模块提供了基于地形四叉树的地形数据管理框架,系统中曾实现一部分地形编辑功能。但在之前的实现中,地形编辑器与虚拟现实引擎分离,编辑器采用了与引擎不同的数据结构。编辑时需要固定编辑范围和编辑层级,只能在较小的范围内进行编辑,编辑结束后才能在引擎中进行浏览和漫游。作为之前工作的延续,本文的工作整合和扩展了编辑功能,在引擎中实现高程和纹理编辑,支持了在任意层级和视角进行编辑,方便用户编辑出不同的尺度的地形特征,并在编辑的同时进行漫游。在原有的实现中,进入编辑状态后就固定了地形块的层级,将待编辑地形作为单层地形看待。在基于地形四叉树实现编辑、保存等操作时,则需要考虑编辑结果向多个层级的同步,和编辑效果的实时呈现。这增加了地形编辑器正确、高效的实现的难度。同时,地形编辑往往难以同时兼顾大规模和丰富的细节,地形编辑器的实现目标应当是用更少的编辑代价编辑出更大范围的真实地形。针对上述问题,本文工作围绕大规模场景的地形编辑进行。
\section{本文主要内容}
本文设计并实现大规模场景的地形编辑器,着重解决在地形高程和纹理编辑中的效果和效率问题。本文的主要内容包括:\par
(一)基于地形四叉树数据结构,对地形高程和纹理的实时编辑框架进行设计和实现。该框架应能有效的对地形数据进行管理,使数据调度可以满足实时编辑需要,并对编辑操作在地形四叉树不同层上的同步、撤销和重做、编辑工程文件的保存和加载、编辑结果的导出等功能提出解决方案。对于地形高程,可以通过笔刷、选区等工具进行升降、平滑、腐蚀、平整等操作。对于地面纹理,可以使用笔刷工具,在多个图层上用不同纹理材质进行涂抹。\par
(二)提出一种基于地形高程规则的纹理笔刷,通过调整自定义规则中的高度区间、斜率区间、过渡带宽度等参数,定制符合真实世界地物分布规律的笔刷规则。用户可以在每一笔编辑之前灵活的选择是否应用规则,并用简单的涂抹快速构建出具有真实感的地表纹理。\par
(三)实现一种过程式地面纹理渲染技术,通过贴花技术和颜色提取技术在实时渲染中以较低的开销为地面纹理添加丰富细节,并通过对夏季卫星影像进行色调调整,实现了不同季节的效果。\par
(四)提出一种定义地形语义信息的方法,实现基于高程和纹理数据的语义数据生成,支持对语义的可视化和编辑。最后给出使用语义信息渲染海洋的实例。\par
\section{本文结构}
本文组织结构如下:\par 
第一章为引言,介绍大规模场景地形编辑器的研究背景、主要问题、研究意义及文章的主要内容。\par 
第二章为相关工作综述,对大规模地形场景生成相关领域的已有工作和技术进行了简要介绍。\par 
第三章介绍了本文所设计和实现的地形编辑器的架构,对编辑器中地形数据的管理调度、编辑流程、编辑操作撤销和重做、编辑结果保存及加载等方面的问题及解决方案进行了阐述。\par
第四章介绍了地形高程编辑的实现,包括基于笔刷、选区等交互形式实现的地形升降、平滑、平整等操作,及地形高程编辑的中间结果的保存和加载。\par
第五章介绍了地形纹理编辑的实现,包括基于笔刷在多个图层上进行纹理编辑,提出了基于高程规则的纹理画刷,使地表纹理编辑可以符合地形高程特征。使用贴花技术实现了基于粗糙卫星影像的纹理细节增强,提升了近地漫游时的真实感。提出了一种方法对卫星影像进行色调调整,以呈现不同季节的地表纹理效果。\par
第六章介绍了本文对地形语义信息的定义,提出了基于已有的高程和纹理数据生成语义数据的方法。基于纹理编辑框架支持了语义编辑,提供了语义信息的可视化,并给出了一个使用语义信息指导渲染的实例。\par
第七章对本文工作进行了总结,提出了进一步研究的方向。

%\pkuthssffaq % 中文测试文字。

% vim:ts=4:sw=4
